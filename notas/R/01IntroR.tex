% Options for packages loaded elsewhere
\PassOptionsToPackage{unicode}{hyperref}
\PassOptionsToPackage{hyphens}{url}
%
\documentclass[
]{article}
\usepackage{amsmath,amssymb}
\usepackage{lmodern}
\usepackage{iftex}
\ifPDFTeX
  \usepackage[T1]{fontenc}
  \usepackage[utf8]{inputenc}
  \usepackage{textcomp} % provide euro and other symbols
\else % if luatex or xetex
  \usepackage{unicode-math}
  \defaultfontfeatures{Scale=MatchLowercase}
  \defaultfontfeatures[\rmfamily]{Ligatures=TeX,Scale=1}
\fi
% Use upquote if available, for straight quotes in verbatim environments
\IfFileExists{upquote.sty}{\usepackage{upquote}}{}
\IfFileExists{microtype.sty}{% use microtype if available
  \usepackage[]{microtype}
  \UseMicrotypeSet[protrusion]{basicmath} % disable protrusion for tt fonts
}{}
\makeatletter
\@ifundefined{KOMAClassName}{% if non-KOMA class
  \IfFileExists{parskip.sty}{%
    \usepackage{parskip}
  }{% else
    \setlength{\parindent}{0pt}
    \setlength{\parskip}{6pt plus 2pt minus 1pt}}
}{% if KOMA class
  \KOMAoptions{parskip=half}}
\makeatother
\usepackage{xcolor}
\usepackage[margin=1in]{geometry}
\usepackage{graphicx}
\makeatletter
\def\maxwidth{\ifdim\Gin@nat@width>\linewidth\linewidth\else\Gin@nat@width\fi}
\def\maxheight{\ifdim\Gin@nat@height>\textheight\textheight\else\Gin@nat@height\fi}
\makeatother
% Scale images if necessary, so that they will not overflow the page
% margins by default, and it is still possible to overwrite the defaults
% using explicit options in \includegraphics[width, height, ...]{}
\setkeys{Gin}{width=\maxwidth,height=\maxheight,keepaspectratio}
% Set default figure placement to htbp
\makeatletter
\def\fps@figure{htbp}
\makeatother
\setlength{\emergencystretch}{3em} % prevent overfull lines
\providecommand{\tightlist}{%
  \setlength{\itemsep}{0pt}\setlength{\parskip}{0pt}}
\setcounter{secnumdepth}{-\maxdimen} % remove section numbering
\ifLuaTeX
  \usepackage{selnolig}  % disable illegal ligatures
\fi
\IfFileExists{bookmark.sty}{\usepackage{bookmark}}{\usepackage{hyperref}}
\IfFileExists{xurl.sty}{\usepackage{xurl}}{} % add URL line breaks if available
\urlstyle{same} % disable monospaced font for URLs
\hypersetup{
  pdftitle={Introducción a R},
  hidelinks,
  pdfcreator={LaTeX via pandoc}}

\title{Introducción a R}
\author{}
\date{\vspace{-2.5em}}

\begin{document}
\maketitle

{
\setcounter{tocdepth}{2}
\tableofcontents
}
\hypertarget{r}{%
\section{R}\label{r}}

\hypertarget{quuxe9-es-r}{%
\subsection{¿Qué es R?}\label{quuxe9-es-r}}

\texttt{R} es un lenguaje de programación originado en 1993, es un
proyecto GNU que es un dialecto de su antecesor \texttt{S}, un proyecto
de investigación en los laboratorios Bell por ahí de los 80's. Hoy en
día \texttt{R} es de los lenguajes de programación más importantes en el
área de ciencia de datos.

\texttt{R} es un lenguaje de programación enfocado en el análisis
estadístico y gráfico; se puede usar en diferentes plataformas y
sistemas operativos. Es mantenido por un equipo de personas, y está
soportado por una comunidad impresionante.

\emph{TL;DR}: \texttt{R} es un lenguaje libre que nos ayuda a leer,
transformar, graficar, modelar y entender datos.

\hypertarget{por-quuxe9-r}{%
\subsection{¿Por qué R?}\label{por-quuxe9-r}}

\begin{itemize}
\tightlist
\item
  Está enfocado al manejo modelado y graficación de información.
\item
  Es libre y gratuito.
\item
  La comunidad que lo soporta es lo que le da tanto valor.
\item
  Miles de paquetes, con los últimos desarrollos en estadística.
\item
  Manejo de series de tiempo, bases de datos, datos geoespaciales.
\item
  Conecta muy bien con todo tipo de sistemas.
\end{itemize}

Una de las características centrales del lenguaje, que en mi opinión son
fundamentales para cuando vamos empezando a programar, es el papel tan
central que tiene un el \texttt{data.frame} en el lenguaje.

Python por ejemplo es un lenguaje de programación enorme y muy complejo,
bastante útill; sin embargo es muy fácilperderse en todo lo que python
tiene para ofrecer.

Por otro lado \texttt{R} tiene un ambiente que se asemeja más a una hoja
de cálculo, el data frame nos ayuda a pensar en tablas y no sólo en
clases o toda la complejidad inherente a un lenguaje de programación.

\hypertarget{por-quuxe9-aprender-a-programar}{%
\section{¿Por qué aprender a
programar?}\label{por-quuxe9-aprender-a-programar}}

Una analogía a por qué aprender a programar \texttt{R} contra todos los
demás softwares de cómputo, es la siguiente: Usar SPSS (o software
similar), es como ir en un autobús, mientras que aprender a programar en
\texttt{R} es como manejar tu mismo un auto.

La diferencia es evidente y la flexibilidad que te da el poder usar
todas las herramientas, es fantástica. Espero que este módulo les ayude
a construír una caja de herramientas más sólida para cuando lo
necesiten.

Bienvenidos. :)

\end{document}
